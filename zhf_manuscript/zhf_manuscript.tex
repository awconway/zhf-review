% !TeX program = pdfLaTeX
\documentclass[smallextended]{svjour3}       % onecolumn (second format)
%\documentclass[twocolumn]{svjour3}          % twocolumn
%
\smartqed  % flush right qed marks, e.g. at end of proof
%
\usepackage{amsmath}
\usepackage{graphicx}
\usepackage[utf8]{inputenc}

\usepackage[hyphens]{url} % not crucial - just used below for the URL
\usepackage{hyperref}
\providecommand{\tightlist}{%
  \setlength{\itemsep}{0pt}\setlength{\parskip}{0pt}}

%
% \usepackage{mathptmx}      % use Times fonts if available on your TeX system
%
% insert here the call for the packages your document requires
%\usepackage{latexsym}
% etc.
%
% please place your own definitions here and don't use \def but
% \newcommand{}{}
%
% Insert the name of "your journal" with
% \journalname{myjournal}
%

%% load any required packages here




\begin{document}

\title{Accuracy and precision of zero-heat-flux temperature monitoring: a
systematic review and meta- analysis \thanks{Grants or other notes about the article that should go on the front page
should be placed here. General acknowledgments should be placed at the
end of the article.} }


    \titlerunning{Short form of title (if too long for head)}

\author{  Aaron Conway \and  Âuthóř 2 \and  }

    \authorrunning{ Conway, }

\institute{
        Aaron Conway \at
     Peter Munk Cardiac Centre, UHN \& Lawrence S. Bloomberg Faculty of
 Nursing, University of Toronto \\
     \email{\href{mailto:aaron.conway@utoronto.ca}{\nolinkurl{aaron.conway@utoronto.ca}}}  %  \\
%             \emph{Present address:} of F. Author  %  if needed
    \and
        Âuthóř 2 \at
     Department of ZZZ, University of WWW \\
     \email{\href{mailto:djf@wef}{\nolinkurl{djf@wef}}}  %  \\
%             \emph{Present address:} of F. Author  %  if needed
    \and
    }

\date{Received: date / Accepted: date}
% The correct dates will be entered by the editor


\maketitle

\begin{abstract}
The text of your abstract. 150 -- 250 words.
\\
\keywords{
        key \and
        dictionary \and
        word \and
    }

    \subclass{
                    MSC code 1 \and
                    MSC code 2 \and
            }

\end{abstract}


\def\spacingset#1{\renewcommand{\baselinestretch}%
{#1}\small\normalsize} \spacingset{1}


\hypertarget{introduction}{%
\section{Introduction}\label{introduction}}

Core temperature monitoring is an important intervention within
perioperative and intensive care settings\((ref)\). Thermoregulatory
dysfunction is commonly associated with the induction of anesthesia and
can lead to adverse outcomes including cardiac dysrhythmias, altered
hemostasis and increased risk of surgical site infection (Frank et al.
1997; Kurz, Sessler, and Lenhardt 1996; Michelson et al. 1994; Rohrer
and Natale 1992). The pulmonary artery catheter is the reference
standard for continuous core temperature monitoring\((ref)\), however,
the invasive nature of this method renders an increased risk of
bloodstream infections and damage to surrounding tissue(Hadian and
Pinsky 2006). Although clinically accurate relative to pulmonary artery
thermometry, surrogate measures of core temperature via esophageal,
rectal, and bladder sites remain mildly invasive interventions\((ref)\).

Zero-heat-flux (ZHF) thermometry is an alternative non-invasive method
that allows for continuous monitoring of core temperature. Originally
developed in the 1970s, ZHF technology was recently implemented in the
3M SpotOn (3M, St Paul, MN), a single-use, disposable sensor \((ref)\).
In practice, the ZHF sensor, such as the 3M SpotOn, is placed on the
lateral surface of the forehead and initially warmed to equilibrate the
temperature of the skin surface to the underlaying core
tissues\((ref)\). Equipped with a thermal insulator, the ZHF sensor
eliminates heat loss to the environment
\((thereby creating an isothermal pathway between the skin surface and underlaying core tissues)\)
to allow for changes in core temperature to be directly reflected by a
change in skin surface temperature \((ref)\).

The agreement between ZHF and established core temperature measures has
been investigated in several studies conducted within the clinical
setting, each with varied results. Appraisal of these studies and
synthesis of the results would aid
\(clinical decision-making regarding the appropriate circumstances in which ZHF monitoring may be used.\)

We aimed to determine if the ZHF sensor has clinically acceptable
accuracy and precision compared with established core temperature tools.
Accuracy is defined as the average difference between the ZHF sensor and
established core temperature instrument and precisions as the variance
(typically reported as the SD) in the differences.

\hypertarget{methods}{%
\section{Methods}\label{methods}}

A systematic review was conducted in accordance with a predetermined
protocol (PROSPERO trial registration number:
\#\#\#\#\#\#\#\#\#\#\#\#\#\#\#\#).

\hypertarget{data-sources-and-searches}{%
\subsection{Data sources and searches}\label{data-sources-and-searches}}

Published studies were found by searching Medline and EMBASE from
January 2000 to July 2019,
\(as well as the reference lists of articles identified to be relevant to the review. This search strategy is an efficient approach for systematic reviews of diagnostic test accuracy studies [@preston2015improving]. Unpublished and ongoing studies were located by searching the International Clinical Trials Platform.\)
Published conference abstracts were included if there was enough
information reported to appraise the quality of the study. There were no
language restrictions applied during the search.
\(The Cochrane-recommended search strategy combining terms for the ‘target condition’ and ‘index test’ was used [@mann2012should]\).
The search strategies used for each data base can be located in
\href{}{file 1}.

The selection of studies was undertaken by two independent reviewers.
Studies were included if reported measures from a ZHF sensor and another
established core temperature tool coincided. Studies involving a case
control design were excluded due to potential for overestimation of the
intervention performance. Studies were excluded if conducted on
non-human subjects or outside of a clinical healthcare setting. Studies
were excluded if conducted earlier than the year 2000 as prior studies
evaluated outdated technology that is no longer relevant to current
clinical practice.

\hypertarget{data-extraction-and-quality-assessment}{%
\subsection{Data extraction and quality
assessment}\label{data-extraction-and-quality-assessment}}

\(Information was extracted regarding study characteristics(author, year of publication, country, design, sample size, clinical setting, numbers studied and analyses for each outcome), population characteristics (inclusion and exclusion criteria) and ZHF characteristics (timing and methods of measurements, and methods of calibration).\)
The outcomes that were extracted included the mean bias (eg, accuracy)
and variance (eg, SD, precision) in temperature between the ZHF sensor
and established core temperature measurement tool.
\(Information was extracted about how repeated measurements were handled: (1) analysed each pair of data separately; (2) treated each pair of data as independent; or (3) used either analysis of variance or a random effects model as a way to control for the dependent nature of the repeated measures data [@myles2007using].\)

Using the revised Quality Assessment of Diagnostic Accuracy Studies
(QUADAS-2), each reviewer independently assessed the risk of bias for
the included studies (Whiting et al. 2011).
\(Reviewers used guiding questions to rate the risk of bias for patient selection\),
conduct of the ZHF measurements, conduct of the established core
temperature measurement tools, and timing and flow (eg, timing of ZHF
and established core temperature measurements, dropouts) as ``high'',
``low'' or ``unclear''. We worked to minimize the risk of publication
bias by conducting a comprehensive search of multiple databases as well
as an international clinical trial registry (Glasziou et al. 2001).
\(Statistical analyses to detect reporting bias were not conducted due to lack of validated methods [@begg2005systematic]. Although some meta-analyses of method comparison studies have used tests for detecting funnel plot asymmetry [@niven2015accuracy], simulations have revealed that such tests will result in publication bias being incorrectly identified too often [@deeks2005performance]\).
In order to rate the quality of evidence, we applied the Grading Quality
of Evidence and Strength of Recommendation methodology (Schünemann et
al. 2008). Evidence was downgraded in accordance with study limitations,
inconsistency and imprecision. There were no circumstances in which
evidence was downgraded for indirectness as this systematic review only
included relevant studies. Although the possibility of publication bias
was not excluded, this bias was not formally assessed as it was not
considered sufficient enough to reason downgrading the quality of
evidence.

\hypertarget{data-synthesis-and-analysis}{%
\subsection{Data synthesis and
analysis}\label{data-synthesis-and-analysis}}

The objective
\(for the meta-analysis was to estimate the population limits of agreement between\)
temperature measurements by the ZHF sensor and established core
temperature measurement tools.
\(The framework for meta-analysis of Bland-Altman method comparison studies based on limits of agreement (LoA) approach was used [@tipton2017framework]\).
This method was selected because it parallels the approach used in
primary Bland-Altman studies, whereby an estimate is generated for the
pooled LoAs in the population (not just in the samples studied). The
`population LoA' is more broad than the LoA commonly reported in the
meta-analyses of Bland-Altman studies (Tipton and Shuster 2017). In
these instances, \$the pooled LoAs are calculated using
\(\delta\pm2\sqrt{\sigma2+\tau2}\), where \(\delta\) is the average bias
across studies, \(\sigma2\) is the average within-study variation in
differences and \(\tau2\) is the variation in bias across studies.\$
Estimations of \(\delta\) and \(\sigma2\) were made using a weighted
least-squares model (similar to a random-effects approach), with the
associated SEs estimated using robust variance estimation (RVE). RVE was
used alternatively to model-based SEs as several studies included in the
systematic review used repeated-measures designs without accommodating
for the correlation between measurements (Hedges, Tipton, and Johnson
2010; Tanner-Smith, Tipton, and Polanin 2016; Tipton 2015). We used the
\$method-of-moments estimator from ref 16 for the \(\tau2\) parameter\$.
In accordance with ref 12, (1) measures of uncertainty were included
\$when interpreting the LoA estimates by calculating the outer \(95\%\)
CIs for pooled LoA\$; and (2) repeated measurements were adjusted if
improperly adjusted for in
\(individual studies (by using weights proportional to the number of samples, not the total number of measurements)\).
The formulas for the above calculations from ref 12 can be found in
\href{}{file 1}. The R statistical program was use to conduct all
analyses (Team 2017). All data and R code (provided in ref 12) used in
the meta-analyses can be located \href{https://doi.org}{here}.

Prior to conducting the meta-analyses, the results from each study were
converted into a standard format, with bias meaning \(ECT - ZHF\)
measured in \(^\circ C\). In two of the studies, the results were
reported for two separate groups of participants, so these were treated
in the meta-analyses as separate `studies'.
\(Other studies reported separate results for analyses conducted using different TcCO2 device types or sensor locations performed on the same patients. Only the result with the largest number of paired measurements between PaCO2 and TcCO2 was selected for inclusion in the main analysis, with others included in subgroup meta-analyses where appropriate. For the studies that reported results for patients while receiving both two-lung and one-lung ventilation during thoracic surgery, we used the result for two-lung ventilation in the main analysis. The conventionally cited clinically acceptable agreement between\)
ECT and ZHF sensors is 0.5 \(^\circ C\)(ref). \$It was deemed that outer
confidence bounds for \(95\%\) LoA between ECT and ZHF (termed as
`population limits of agreement') outside of these bounds would\$ be
clinically unacceptable. Sensitivity analysis for the primary
meta-analysis was performed based on risk of bias, whereby `unclear risk
of bias' was treated as `high risk' and `high risk of bias' studies were
excluded from analyses.
\(As clinicians would be interested in the accuracy of\) ZHF relative to
temperature monitoring devices they use, and within the clinical setting
in which they use it we conducted subgroup analyses according to the
comparator device used (eg, core, peripheral, nasopharangeal) and
clinical setting (eg, intraoperative, intensuve care unit).

\hypertarget{results}{%
\section{Results}\label{results}}

\begin{longtable}[]{@{}lllll@{}}
\toprule
test1 & test2 & test3 & test4 & test5\tabularnewline
\midrule
\endhead
a & b & c & d & e\tabularnewline
a & b & c & d & e\tabularnewline
a & b & c & d & e\tabularnewline
\bottomrule
\end{longtable}

\hypertarget{study-selection-and-description}{%
\subsection{Study selection and
description}\label{study-selection-and-description}}

\hypertarget{agreement-between-zhf-and-core-temperature-measurements}{%
\subsection{Agreement between ZHF and core temperature
measurements}\label{agreement-between-zhf-and-core-temperature-measurements}}

\hypertarget{discussion}{%
\section{Discussion}\label{discussion}}

\hypertarget{limitations}{%
\subsection{Limitations}\label{limitations}}

\hypertarget{conclusion}{%
\section{Conclusion}\label{conclusion}}

\hypertarget{references}{%
\section*{References}\label{references}}
\addcontentsline{toc}{section}{References}

\hypertarget{refs}{}
\leavevmode\hypertarget{ref-frank1997perioperative}{}%
Frank, Steven M, Lee A Fleisher, Michael J Breslow, Michael S Higgins,
Krista F Olson, Susan Kelly, and Charles Beattie. 1997. ``Perioperative
Maintenance of Normothermia Reduces the Incidence of Morbid Cardiac
Events: A Randomized Clinical Trial.'' \emph{Jama} 277 (14). American
Medical Association: 1127--34.

\leavevmode\hypertarget{ref-glasziou2001systematic}{}%
Glasziou, Paul, Les Irwig, Chris Bain, and Graham Colditz. 2001.
\emph{Systematic Reviews in Health Care: A Practical Guide}. Cambridge
University Press.

\leavevmode\hypertarget{ref-hadian2006evidence}{}%
Hadian, Mehrnaz, and Michael R Pinsky. 2006. ``Evidence-Based Review of
the Use of the Pulmonary Artery Catheter: Impact Data and
Complications.'' \emph{Critical Care} 10 (3). BioMed Central: S8.

\leavevmode\hypertarget{ref-hedges2010robust}{}%
Hedges, Larry V, Elizabeth Tipton, and Matthew C Johnson. 2010. ``Robust
Variance Estimation in Meta-Regression with Dependent Effect Size
Estimates.'' \emph{Research Synthesis Methods} 1 (1). Wiley Online
Library: 39--65.

\leavevmode\hypertarget{ref-kurz1996perioperative}{}%
Kurz, Andrea, Daniel I Sessler, and Rainer Lenhardt. 1996.
``Perioperative Normothermia to Reduce the Incidence of Surgical-Wound
Infection and Shorten Hospitalization.'' \emph{New England Journal of
Medicine} 334 (19). Mass Medical Soc: 1209--16.

\leavevmode\hypertarget{ref-michelson1994reversible}{}%
Michelson, Alan D, Hollace MacGregor, Marc R Barnard, Anita S Kestin,
Michael J Rohrer, and C Robert Valeri. 1994. ``Reversible Inhibition of
Human Platelet Activation by Hypothermia in Vivo and in Vitro.''
\emph{Thrombosis and Haemostasis} 72 (05). Schattauer GmbH: 633--40.

\leavevmode\hypertarget{ref-rohrer1992effect}{}%
Rohrer, Michael J, and ANITA M Natale. 1992. ``Effect of Hypothermia on
the Coagulation Cascade.'' \emph{Critical Care Medicine} 20 (10):
1402--5.

\leavevmode\hypertarget{ref-schunemann2008grading}{}%
Schünemann, Holger J, Andrew D Oxman, Jan Brozek, Paul Glasziou, Roman
Jaeschke, Gunn E Vist, John W Williams, et al. 2008. ``Grading Quality
of Evidence and Strength of Recommendations for Diagnostic Tests and
Strategies.'' \emph{Bmj} 336 (7653). British Medical Journal Publishing
Group: 1106--10.

\leavevmode\hypertarget{ref-tanner2016handling}{}%
Tanner-Smith, Emily E, Elizabeth Tipton, and Joshua R Polanin. 2016.
``Handling Complex Meta-Analytic Data Structures Using Robust Variance
Estimates: A Tutorial in R.'' \emph{Journal of Developmental and
Life-Course Criminology} 2 (1). Springer: 85--112.

\leavevmode\hypertarget{ref-team2017r}{}%
Team, R Core. 2017. ``R Core Team (2017). R: A Language and Environment
for Statistical Computing.'' \emph{R Found. Stat. Comput. Vienna,
Austria. URL Http://Www. R-Project. Org/., Page R Foundation for
Statistical Computing}.

\leavevmode\hypertarget{ref-tipton2015small}{}%
Tipton, Elizabeth. 2015. ``Small Sample Adjustments for Robust Variance
Estimation with Meta-Regression.'' \emph{Psychological Methods} 20 (3).
American Psychological Association: 375.

\leavevmode\hypertarget{ref-tipton2017framework}{}%
Tipton, Elizabeth, and Jonathan Shuster. 2017. ``A Framework for the
Meta-Analysis of Bland--Altman Studies Based on a Limits of Agreement
Approach.'' \emph{Statistics in Medicine} 36 (23). Wiley Online Library:
3621--35.

\leavevmode\hypertarget{ref-whiting2011quadas}{}%
Whiting, Penny F, Anne WS Rutjes, Marie E Westwood, Susan Mallett,
Jonathan J Deeks, Johannes B Reitsma, Mariska MG Leeflang, Jonathan AC
Sterne, and Patrick MM Bossuyt. 2011. ``QUADAS-2: A Revised Tool for the
Quality Assessment of Diagnostic Accuracy Studies.'' \emph{Annals of
Internal Medicine} 155 (8). Am Coll Physicians: 529--36.

\bibliographystyle{spbasic}
\bibliography{bibliography.bib}

\end{document}
